\section{Introduction}
L'objectif de ce rapport est d'analyser la commandabilité du Moteur DC à aimants permanents. Cela s'est fait en commençant par une simulation du système à régler avec MATLAB permettant d'analyser sont comportement. Une identification du système par plusieurs méthodes permet de retrouver les paramètres décrivant le système à régler pour configurer le régulateur. \bigskip 


\par Puis la commande sera dimensionnée et testée en la programmant en C afin de pouvoir utiliser le régulateur en pratique. Finalement quelques tests seront fait afin de valider les résultats obtenus en simulation et pour observer le comportement du système pour différentes configurations (Système avec/ou sans anti-windup, système avec le rotor bloqué ou non), déterminer leur efficacité et choisir la meilleure.

%\begin{figure}[H]
%	\begin{center}
% 		\includegraphics [width=8cm]{systeme_a_carateriser.png}
%		\includegraphics [width=8cm]{Image1.png}
%	\end{center}
% 	\caption{Gauche, système à caractériser - Droite, mécanisme du système}
% 	\label{}
%\end{figure}